\documentclass[12pt, letterpaper, twoside]{article}
\usepackage[utf8]{inputenc}
\usepackage{graphicx}
\begin{document}
\title{IN2090 Oblig 3}
\author{Espen Lønes}
\date{\today}
\maketitle
\ \\
Oppgave 1)\\
\begin{verbatim}
CREATE TABLE Tog (
  togNr int PRIMARY KEY,
  startStasjon text NOT NULL,
  endeStasjon text NOT NULL,
  ankomstTid time(0) NOT NULL
);

CREATE TABLE TogTabell (
  togNr int REFERENCES Tog (togNr), 
  avgangsTid time(0) NOT NULL, 
  stasjon text NOT NULL,
  CONSTRAINT tog_tabell_pk PRIMARY KEY (togNr, avgangsTid)
);

CREATE TABLE Plass (
  dato date, 
  togNr int REFERENCES Tog (togNr), 
  vognNr int,
  plassNr int,
  vindu boolean NOT NULL,
  ledig boolean NOT NULL,
  CONSTRAINT plass_pk PRIMARY KEY (dato, togNr, vognNr, plassNr)
);
\end{verbatim}
\newpage
\ \\
Oppgave 2)\\
\ \\
a)\\
\\
Vi har:
\begin{verbatim}
R(A,B,C,D,E,F,G)
1. CDE -> B
2.  AF -> B
3.   B -> A
4. BCF -> DE
5.   D -> G
\end{verbatim}
Finner kandidatnøkkler.\\
Må inneholde de som kun er på venstre sider av FD-er. Dette er C og F.\\
Kan ikke inneholde de som er kun høyre sider av FD-er. Dette er G.\\
Ved å prøve med alle mulige kombinasjoner av CF og de resterende atributter (ABDE). Fant jeg disse kandidatnøkklene.
\begin{verbatim}
{C,F,A}
{C,F,B}
{C,F,D,E}
\end{verbatim}
b)\\ 
\ \\
Den høyeste normalformen som R tilfrestiller er den høyeste normalformen som alle FD-ene til R tilfrestiller. Finner da normalformene til FD-ene.
\begin{verbatim}
R(A,B,C,D,E,F,G)
1. CDE -> B   3NF 
2.  AF -> B   3NF
3.   B -> A   3NF
4. BCF -> DE  BCNF
5.   D -> G   1NF
\end{verbatim}
\newpage
\ \\
1. Fordi CDE ikke er en supernøkkel men B er en nøkkelattributt.\\
2. Fordi AF ikke er en supernøkkel men B er en nøkkelattributt.\\
3. Fordi B ikke er en supernøkkel men A er en nøkkelattributt.\\
4. Fordi BCF er en supernøkkel.\\
5. Fordi D ikke er en supernøkkel, G er ikke en nøkkelattributt og D er en del av en kandidatnøkkel.\\
\\
Så R er på 1NF.\\
\ \\
c)\\
\ \\
Så vi har:
\begin{verbatim}
R(A,B,C,D,E,F,G)
FD-er:
1. CDE -> B
2.  AF -> B 
3.   B -> A   
4. BCF -> DE 
5.   D -> G 
Kandidatnøkler:
{A,C,F}
{B,C,F}
{C,D,E,F}
\end{verbatim}
Som vi vet fra oppgave b) så bryter FD 1 med BCNF. Dekomponerer da R til;
$$
S_1((CDE)^+) = S_1(A,B,C,D,E,G)
$$
og
$$
S_2((CDE),ABCDEFG/(CDE)^+) = S_2(C,D,E,F)
$$
\ \\
Ser så videre på de to komponentene. Der jeg ser på $S_1$ først. 
\newpage
\ \\
\begin{verbatim}
S_1(A,B,C,D,E,G)
FD-er:
1. CDE -> B
2.   B -> A
3.   D -> G
Kandidatnøkler:
{C,D,E}
\end{verbatim}
\ \\
1. FD er på BCNF fordi CDE er en supernøkkel.\\
Men 2. FD er ikke det fordi B ikke er en supernøkkel.\\
Så vi dekomponerer $S_1$, som gir:\\
$$
S_{11}(B^+) = S_{11}(A,B)
$$
og
$$
S_{12}(B,(ABCDEG)/B^+) = S_{12}(B,C,D,E,G)
$$
\ \\
Ser så på $S_{11}$
\begin{verbatim}
S_11(A,B)
FD-er:
1.   B -> A
Kandidatnøkler:
{B}
\end{verbatim}
FD-en tilfredstiller BCNF så vi trenger ikke dekomponere $S_{11}$
\newpage
\ \\
Ser så på $S_{12}$\\
\begin{verbatim}
S_12(B,C,D,E,G)
FD-er:
1. CDE -> B
2.   D -> G
Kandidatnøkler:
{C,D,E}
\end{verbatim}
1. FD er på BCNF siden CDE er en supernøkkel.\\
Men 2. FD er ikkedet siden D ikke er en supernøkkel.\\
Dekomponerer da $S_{12}$:
$$
S_{121}(D^+) = S_{121}(D,G)
$$
og
$$
S_{122}(D,(BCDEG)/B^+) = S_{122}(B,C,D,E)
$$
Ser på $S_{121}$.
\begin{verbatim}
S_121(D,G)
FD-er:
1.   D -> G
Kandidatnøkler:
{D}
\end{verbatim}
D er en supernøkkel så FD-en er på BCNF og $S_{121}$ trenger ikke dekomponeres videre.\\
\ \\
Ser på $S_{122}$.
\begin{verbatim}
S_122(B,C,D,E)
FD-er:
1. CDE -> B
Kandidatnøkler:
{C,D,E}
\end{verbatim}
CDE er en supernøkkel så FD-en er på BCNF og $S_{122}$ trenger ikke dekomponeres videre.
\newpage
\ \\
Så må vi tilbake og se på $S_2$
\begin{verbatim}
S_2(C,D,E,F)
FD-er:

Kandidatnøkler:

\end{verbatim}
$S_2$ har ingen FD-er og er dermed på BCNF og trenger ikke dekomponeres.\\
\ \\
Oppsumert får vi da at
\begin{verbatim}
R(A,B,C,D,E,F,G)
\end{verbatim}
Dekomponeres til:
\begin{verbatim}
S_11(A,B)
S_121(D,G)
S_122(B,C,D,E)
S_2(C,D,E,F)
\end{verbatim}
\end{document}